\chapter{Introduction}
\label{intro}

The recent discovery of a new boson of mass ${\sim}125$~\GeVcc and the
increasing evidence of its consistency with the Higgs boson are
striking victories (strange wording) for the \acrfull{sm}. With its success in describing
electroweak symmetry breaking, the SM continues as the most
well-tested and robust description of modern particle physics (that we
havew today). However in
passing this important test, the shortcomings of this model become
even more disturbing (pronounced). The SM offers no explanation for the tremendous
amount of dark matter in our universe (capitalize?), nor of the observed disparity
between matter and anti-matter. In particular it suffers from what is
known as the Hierarchy Problem, which posits that if this
newly-discovered boson is exactly as predicted, then
either its mass of ${\sim}125$~\GeVcc is anomalously light or the
result of an extraordinarily coincidental and precise cancellation of
the order of one part in $10^{26}$.

Many~\cite{CMS:2012discovery} of the whatever


\section{Section Durp}
\label{sectiondurp}

This is a reference to Section~\ref{sectiondurp}.