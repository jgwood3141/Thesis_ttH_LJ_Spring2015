\chapter{Conclusions and Summary}
\label{summary_overview}

\par In 2012, the Large Hadron Collider (LHC) produced the highest
energy proton-proton collisions, with center of mass energies of 8
\TeV.  Protons, with humble beginnings in a bottle of Hydrogen, travel
through a multi-stage accelerator complex, before being injected into
the 27.6 km LHC ring as two counter-rotating beams.  Superconducting
radio-frequency cavities accelerate the beams during each revolution,
constrained to the circular path by more than a thousand 8 T
superconducting dipole magnets, as each beam is brought to an energy
of 4 \TeV.  

\par At one of the four points on the LHC ring where the proton beams
are squeezed together to produce collisions, sits the Compact Muon
Solenoid (CMS) experiment, a general purpose particle detector
designed to elucidate the mechanism of electroweak symmetry breaking,
and explore physics interactions at the \TeV energy scale.  This
14,000 ton, 15 m tall device, provides hermetic, 4$\pi$, coverage of
the interaction region, and is composed of a system of sub-detectors,
with a cylindrical symmetry about the beam-line and interaction
region, which work in parallel to identify and measure the kinematic
properties of particles produced during $pp$ collisions.  The inner
tracking system is composed of more than 70 million silicon pixel and
strip detectors that provide $\mu$m spatial resolution on the
trajectory of charged particles.  An electromagnetic calorimeter
(ECAL) surrounds the inner tracker, and is composed of more than
75,000 lead-tungstate crystals, which absorb energy from
electromagnetically interacting particles, with electrons and photons
depositing almost all of their energy in this sub-detector.  The
hadronic calorimeter (HCAL) surrounds the ECAL, absorbing the energy
of charged and neutral hadrons with stacks of brass absorber material
with layers of plastic scintillator to sample the energy in between.
The outermost system are the muon chambers, which utilize three
different types of detector technologies to provide fast timing to
trigger measurements and excellent spatial resolution on muons, an
important signature for many \TeV energy scale processes.  Hardware
and firmware installed on the detector provide an instant, but basic
reconstruction of a collision, allowing for the amount of collisions
recorded to be reduced from a rate of 40 MHz down to 10 kHz.  Events
are additionally filtered through the use of software to a manageable
rate of 100 Hz.  

\par Once the first 5.1 \fbinv of 8 \TeV data was collected by the CMS
detector, a search for the Standard Model Higgs boson, produced in
association with top-quark pairs (\ttH) was performed in the final
state with a single lepton, at least 4 jets and at least 2 $b$-tags.
The search region was divided into categories based on the jet and
$b$-tag multiplicity of the final state, and a Clermond-Ferrand
Multi-Layer Perceptron Artificial Neural Network (CFMlpANN) was
trained to provide a one-dimensional discriminant for how likely the
event is to be from the \ttH signal, or one of the \ttjets
backgrounds.  No significant excess of events in the data was
observed, and an observed (expected) upper limit on the production
rate of \ttH at 9.5 (5.4) times the rate predicted by the Standard
Model.  This final state was combined with a di-lepton final state,
and the previous results from the 7 \TeV dataset collected in 2011,
produce an observed (expected) upper limit on the \ttH process as 5.8
(5.2) times the Standard Model rate and published in the Journal of
High Energy Physics (JHEP) in May of 2013.   

\par A second analysis was performed on the full 19.5 \fbinv dataset of
8 \TeV data collected by CMS.  This also used a final state with a
single lepton, at least 4 jets, and at least 2 $b$-tags, and a search
region divided into categories based on the jet and $b$-tag
multiplicity of the final state.  A different multivariate analysis
(MVA) technique was employed: a Boosted Decision Tree (BDT) was
trained to separate the \ttH signal from the \ttjets background for
each of the jet/tag categorizations.  Once again, no significant
excess of events is observed, and an observed (expected) upper limit
on the \ttH production rate is set at 4.9 (4.7) times the Standard
Model prediction.  This analysis was combined with same and opposite
sign di-lepton, mutli-lepton, and hadronic tau final states to produce
an observed (expected) upper limit of 4.5 (2.5) time the predicted rate
of \ttH production.  

\par In preparation to perform this search in the next dataset
collected by CMS, several investigations have been performed on ways
to improve the sensitivity of the analysis to the \ttH signal.  One of
the most important improvements will be the incorporation of
next-to-leading order (NLO) QCD effects into the simulation of \ttH
signal and \ttjets background.  This will improve the modeling of high
jet-multiplicity events, which characterize both the signal and
background in this analysis.  These improved simulations will also
incorporate the latest techniques to calculate the spin-correlations
of the decay products from heavy resonances in top-quark and $W$ boson
decays, via the MadSpin framework.  This will allow the angular
correlations of the daughters of the \ttbar system to be used to
correctly associate jets in an event to their roles in the \ttbar
decay, thereby reducing the combinatorics of jets that can possibly be
associated with jets from the Higgs decay.  

\par With the experience gained in previous analyses, and improvements
already underway, the observation of a \ttH signal will be
increasingly likely in the larger statistics, higher-energy datasets
collected in the future by \ttH.  In the lack of an observation, now
or in the future, these upper limits can be used to constrain future
models involving physics beyond the Standard Model (BSM) that would
predict enhancements to final states explored in these first two \ttH
analyses.  