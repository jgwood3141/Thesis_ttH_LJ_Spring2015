\chapter{Particle Reconstruction at CMS}
\label{reconstruction_overview}

\par Data is reconstructed at CMS using the $Particle Flow^{TM}$ algorithm

\section{Muon Reconstruction}
\label{muon_reco_overview}

\par Muons rely heavily on the inner tracker and muons chambers for
efficient identrification and reconstruction

\section{Electron Reconstruction}
\label{electron_reco_overview}

\par Electrons leave charged tracks in the inner tracker, and create a
wide shower of particles and thus energy deposits in the ECAL.  High
energy electrons sometimes traverse the entire distance of the ECAL
and leave energy in the HCAL, however the ratio of these two energies
is disproportionate for the ECAL, and thus this ratio is often used to
discriminate electrons from highly electromagnetic hadronic jets.

\section{Photon Reconstruction}
\label{photon_reco_overview}

\par Like electrons, but with no tracks, and narrower shower shape.

\section{Jet Reconstruction}
\label{jet_reco_overview}

\par Jets are formed by matching tracks from the inner tracker to
energy deposits in the ECAL and HCAL.  Energy clusters are identified
from the ECAL and HCAL, and everything is then clustered in a cone.

\section{Tau Reconstruction}
\label{tau_reco_overview}

\par So heavy that they decay to leptons or hadrons before traversing
the detector, they still leave an oddly-numbered pronged decay hadronically due
to charge conservation requiring that one of the hadrons produced be
eqaul charge to the tau.  This results in one charged, and any number
of neutral pions, or three charged, and any number of neutral pions. 

\section{Missing Transverse Energy Reconstruction}

\par since the detector is hermetic, and the tracker so granular, we
can ensure that no particles flew out of the detector due to lack of
coverage.  Only long-lived neutral particles can escape, such as
neutrinos in the standard model.  Many \acrshort{bsm} theories, such
as SUSY, are characterized by stable, neutral particles. 

\par MET is the vector sum of all of the tracks associated with a
particular primary vertex (? or all vertices in event).  Thus if there
wasa neutral particle that escaped detection, there would be a
momentum imbalance along the trajectory of that particle.  This is how
neutrinos are identified.  
